\documentclass{article}

% set font encoding for PDFLaTeX or XeLaTeX
\usepackage{ifxetex}
\ifxetex
  \usepackage{fontspec}
\else
  \usepackage[T1]{fontenc}
  \usepackage[utf8]{inputenc}
  \usepackage{lmodern}
\fi

% used in maketitle
\title{My first \LaTeX{} document}
\author{My Name}

% Enable SageTeX to run SageMath code right inside this LaTeX file.
% documentation: http://mirrors.ctan.org/macros/latex/contrib/sagetex/sagetexpackage.pdf
\usepackage{sagetex}

\begin{document}

\section{First Steps}

This is a short introduction document to \LaTeX{}.
You write the code on the left hand side of this editor,
while compiled and rendered output shows up on the right hand side.

Pay attention to any syntax errors!
\LaTeX{} is a programming language and stops compiling and updating the document upon errors.
Any command starts with a backslash.
Watch out for the red "Errors" tab and inspect any problems.

\subsection{Subsection}

You can organize your document in sections,
just like this one here is a subsection!

\subsection{Another one ...}

Here is \textit{another one},
where the text ``another one'' is formatted italic.

\textbf{Bold font} is also possible.

\subsection{Formulas}

Latex is famous for setting formulas.
This is usually done between dollar signs.
For example: $\int_{x=0}^{\infty} \frac{1}{2 + x^2}\;\mathrm{d}x$.

\section{SageMath}

You can also run a couple of computations with Sage and embed the output right here in the document.

% this line here is an invisble comment
% and the block below some sage code setting x and y but not producing any output.
\begin{sagesilent}
x = var('x')
a = 1328782374
b = 2394728347628374
\end{sagesilent}

% Here, we use x and y defined above:
The product of $\sage{a}$ and $\sage{b}$  is \\
$\sage{a*b}$
and its prime factorization: $\sage{factor(a*b)}$.

It is also possible to create a plot:

\begin{center}
\sageplot[width=.5\textwidth]{plot(sin(x) * cos(3*x), (x, -10, 10))}
\end{center}

\end{document}

